\section{Discussion and Conclusion}
%The results demonstrate that there is clearly a difference in tracking performance
%resulting from the modification to the geometry of the vertex barrels of Sidloi3.
The two vertex detector geometries show differences in tracking performance.
However, the change in performance do not necessarily point to
one geometry being an outright favorite over the other.
In general, the tracking benefits achieved by the vertex barrel modification are
outweighed by some other shortcoming.

For instance, though the modified detector had a higher tracking efficiency for low momentum particles
in both physics processes ($\ee \rightarrow \ttbar$ at $ \sqrt{s} = $ 500 GeV, figure~\ref{fig:eettbareffthetalowpt}; 
$\ee \rightarrow \ttbar \bbbar$ (hadronic decays only) at $ \sqrt{s} = $ 1 TeV, figure~\ref{fig:ttbbeffthetalowpt}),
the modified detector had a higher fake rate for reconstructed tracks (figure~\ref{fig:eettbarfakerate}, figure~\ref{fig:ttbbfakerate}).
Therefore, even though the modified detector will reconstruct a greater number of tracks,
those additional tracks will more often contain mismatched hits from other Monte Carlo particles
than they would have had Sidloi3 reconstructed them because Sidloi3 has a lower fake rate.
Furthermore, the modified detector's higher efficiency for lower momentum particles is again
outweighed by the modified detector's lower efficiency for high momentum particles
(figure~\ref{fig:eettbareffthetahighpt}, figure~\ref{fig:ttbbeffthetahighpt})
and lower efficiency with respect to the number of hits produced the Monte Carlo charged particle
(figure~\ref{fig:eettbareffhit}, figure~\ref{fig:ttbbeffhit}).

In addition, though modified detector
exhibited better $z$-axis impact parameter resolution $\sigma(z_{0})$
for 10 and 100 GeV muons at high polar angles (figure~\ref{fig:muonz0resratio}),
it also demonstrated worse $z$-axis impact parameter resolution
for 1 GeV muons  at lower polar angles (figure~\ref{fig:muonz0resratio})
and worse transverse impact parameter resolution $\sigma(d_{0})$
for 10 and 100 GeV muons at high polar angles (figure~\ref{fig:muond0resratio}).
For the physics processes, though both detectors had
equal transverse impact parameter resolution for a wide polar angle
(figure~\ref{fig:eettbard0resratio}, figure~\ref{fig:ttbbd0resratio}),
the modified detector had worse $z$-axis impact parameter resolution for a wide range
of polar angles (figure~\ref{fig:eettbarz0resratio}, figure~\ref{fig:ttbbz0resratio}).
The modified detector's increased $z$-axis impact parameter resolution 
for high energy single muons (figure~\ref{fig:muonz0resratio})
was present for the two physics processes  (figure~\ref{fig:eettbarz0resratio}, figure~\ref{fig:ttbbz0resratio})
 but to a lesser extent.

The difference in tracking performance between the two vertex detector layouts
 indicates that the modified vertex barrel
geometry is not optimal.
However, it also suggests that the baseline Sidloi3 vertex geometry might not be optimal,
since by some measures the modified detector performed better.% occasionally performed better.
Further studies, with for instance hybrid vertex barrel geometries that mix
doublet tracking and single tracking layers or that have reduced material budgets,
should be conducted to see if it is possible to achieve the improvements demonstrated by
the modified detector without compromising the strong performance of the baseline %Sidloi3
 vertex detector.
Moreover, studies with a different geometry of the outer tracker should also be conducted, so as
%concerning the geometry and digitization of the outer tracker should also be conducted, so as
to pave the road for optimizing the entire tracking system for SiD.

\documentclass[12pt]{article}
\pdfoutput=1
\usepackage{fancyhdr}
\usepackage{authblk}
\usepackage{feynmf}%%%
\usepackage{amsmath,amssymb}
\usepackage{graphicx}
\usepackage{amssymb}
\usepackage{url,color}
\usepackage{epstopdf}
\usepackage{tabularx}
\epstopdfsetup{suffix=}
%\usepackage{multicolumn}
\usepackage{caption}
%\usepackage{subfig}
\usepackage[colorlinks=true, citecolor=green, urlcolor=cyan]{hyperref} 
\usepackage{subcaption}
%\usepackage{siunitx}
\usepackage{textgreek}
\unitlength=1mm
%\usepackage[]{lineno}
%\linenumbers


\textwidth=6.0in  \textheight=8.25in
\renewcommand{\topfraction}{0.99}
\renewcommand{\bottomfraction}{0.8}
%%  Adjust these for your printer:
\leftmargin=-0.3in   \topmargin=-0.20in
\parskip=0.1truein

\newcommand\pubdate{\today}

%%%%%%%%%%%%%%%%%%%%%%%%%%%%%%%%%%%%%%%%%%%%%%%%%%%%%%%%%%%%%%%%%%%%
%%  basic formatting macros:
%%%%%%%%%%%%%%%%%%%%%%%%%%%%%%%%%%%%%%%%%%%%%%%%%%%%%%%%%%%%%%%%%%%
\def\draftnote#1{{\bf [#1]}}
\newcommand{\bspace}{\!\!\!\!}
\def\met{\mbox{$E{\bspace}/_{T}$}}

% Constructing composite symbols -------------------------------------

\def\stacksymbols #1#2#3#4{\def\theguybelow{#2}
    \def\vp{\lower#3pt}
    \def\sp{\baselineskip0pt\lineskip#4pt}
    \mathrel{\mathpalette\intermediary#1}}

\def\intermediary#1#2{\vp\vbox{\sp
     \everycr={}\tabskip0pt
     \halign{$\mathsurround0pt#1\hfil##\hfil$\crcr#2\crcr
              \theguybelow\crcr}}}

%%  single-line equations:

\def\beq{\begin{equation}}
\def\eeq#1{\label{#1}\end{equation}}
\def\eeqn{\end{equation}}
\newcommand{\gsim}{\hbox{ \raise3pt\hbox to 0pt{$>$}\raise-3pt\hbox{$\sim$} }}
\newcommand{\lsim}{\hbox{ \raise3pt\hbox to 0pt{$<$}\raise-3pt\hbox{$\sim$} }}
\newcommand{\mathbold}[1]{\mbox{\boldmath $#1$}}

%%  multiple-line equations  (use \CR as the carriage return):

\newenvironment{Eqnarray}%
   {\arraycolsep 0.14em\begin{eqnarray}}{\end{eqnarray}}
\def\beqa{\begin{Eqnarray}}
\def\eeqa#1{\label{#1}\end{Eqnarray}}
\def\eeqan{\end{Eqnarray}}
\def\CR{\nonumber \\ }

%%  reference to an equation number:

\def\leqn#1{(\ref{#1})}

%%%   special symbols for general use (in math mode)


\def\invfb{ \mbox{fb}^{-1} }
\def\invpb{ \mbox{pb}^{-1} }
\def\roots{ \sqrt{s} }
\def\TeV{ \mbox{TeV}}
\def\GeV{ \mbox{GeV}}
\def\MeV{ \mbox{MeV}}

%
%%   special symbols from the SUSY group

\def\eslt{E_T^{\rm miss}}
\def\ta{\tilde a}
\def\tG{\tilde G}
\def\th{\tilde h}
\def\tH{\tilde H}
\def\tl{\tilde l}
\def\tu{\tilde u}
\def\tc{\tilde c}
\def\ta{\tilde a}
\def\ts{\tilde s}
\def\tb{\tilde b}
\def\tf{\tilde f}
\def\td{\tilde d}
\def\tQ{\tilde Q}
\def\tL{\tilde L}
\def\tH{\tilde H}
\def\tst{\tilde t}
\def\ttau{\tilde \tau}
\def\tmu{\tilde \mu}
\def\tg{\tilde g}
\def\tnu{\tilde\nu}
\def\tell{\tilde\ell}
\def\tq{\tilde q}
\def\tB{\widetilde B}
%\def\tw{\widetilde W}
%\def\tz{\widetilde Z}
\def\alt{\stackrel{<}{\sim}}
\def\agt{\stackrel{>}{\sim}}
\def\be{\begin{equation}}  
\def\ee{\end{equation}}  
\def\bea{\begin{eqnarray}}  
\def\eea{\end{eqnarray}}  
\def\tw{\tilde\chi}
\def\twp{\tilde\chi^+}
\def\twm{\tilde\chi^-}
\def\twpm{\tilde\chi^\pm}
\def\tz{\tilde\chi^0}
\newcommand{\ra}            {\ensuremath{ \rightarrow     }}
\newcommand{\Ra}            {\ensuremath{ \Rightarrow     }}
\newcommand{\longra}        {\ensuremath{ \longrightarrow }}
\newcommand{\Longra}        {\ensuremath{ \Longrightarrow }}
% ----------------------------------------------------------
\newcommand{\la}            {\ensuremath{ \leftarrow      }}
\newcommand{\La}            {\ensuremath{ \Leftarrow      }}
\newcommand{\longla}        {\ensuremath{ \longleftarrow  }}
\newcommand{\Longla}        {\ensuremath{ \Longleftarrow  }}
% ----------------------------------------------------------
\newcommand{\lra}           {\ensuremath{ \leftrightarrow }}
\newcommand{\Lra}           {\ensuremath{ \Leftrightarrow }}
\newcommand{\longlra}       {\ensuremath{ \longleftrightarrow }}
\newcommand{\Longlra}       {\ensuremath{ \Longleftrightarrow }}
\providecommand{\swsqeffl}    {\sin^2\!\theta_{\rm{eff}}^\ell}
\providecommand{\sstw}{\mbox{$\sin^2\theta_W$}}
\providecommand{\MW}      {m_{\mathrm{W}}}
\providecommand{\MZ}      {m_{\mathrm{Z}}}
\providecommand{\MH}      {m_{\mathrm{H}}}
\providecommand{\Mh}      {m_{\mathrm{h}}}
\providecommand{\MT}      {m_{\mathrm{t}}}
\providecommand{\GZ}      {\Gamma_{\mathrm{Z}}}
\providecommand{\GF}         {G_{\mathrm{F}}}
\providecommand{\ppl}  {{\cal P}_{\rm{e}^+}}
\providecommand{\pmi}  {{\cal P}_{\rm{e}^-}}
\providecommand{\ppm}  {{\cal P}_{\rm{e}^\pm}}
\providecommand{\peff}  {{\cal P}_{\rm{eff}}}
\providecommand{\pol}  {{\cal P}}
\providecommand{\Rb}   {R_{\rm{b}}}
\providecommand{\ee}   {\rm{e^+e^-}}
\providecommand{\bb}   {\rm{b\overline{b}}}
\providecommand{\thw}   {\theta_{W}}
\providecommand{\nb}{\,\mathrm{nb}}
\providecommand{\pb}{\,\mathrm{pb}}
\providecommand{\fb}{\,\mathrm{fb}}\providecommand{\pbi}{\,\mathrm{pb}^{-1}}
\providecommand{\fbi}{\,\mathrm{fb}^{-1}}
\providecommand{\Cdgz}{\ensuremath{\Delta g^\mathrm{Z}_1}}
\providecommand{\Cdgg}{\ensuremath{\Delta g^\mathrm{\gamma}_1}}
\providecommand{\Cdkz}{\ensuremath{\Delta \kappa_\mathrm{Z}}}
\providecommand{\Cdkg}{\ensuremath{\Delta \kappa_{\gamma}}}
\providecommand{\Ckg}{\ensuremath{\kappa_{\gamma}}}
\providecommand{\Ckz}{\ensuremath{\kappa_{\mathrm{Z}}}}
\providecommand{\Clg}{\ensuremath{\lambda_{\gamma}}}
\providecommand{\Clz}{\ensuremath{\lambda_{\mathrm{Z}}}}
\providecommand{\Cgv}[1]{\ensuremath{g^V_{#1}}}
\providecommand{\Cgz}[1]{\ensuremath{g^Z_{#1}}}
\providecommand{\Cgg}[1]{\ensuremath{g^{\gamma}_{#1}}}
\providecommand{\Ckzt}{\ensuremath{\tilde{\kappa}_\mathrm{Z}}}
\providecommand{\Clzt}{\ensuremath{\tilde{\lambda}_\mathrm{Z}}}
\providecommand{\Ckgt}{\ensuremath{\tilde{\kappa}_{\gamma}}}
\providecommand{\Clgt}{\ensuremath{\tilde{\lambda}_{\gamma}}}
%
\providecommand{\phmi}{\phantom{-}}
%
\newcommand{\sw}{s_w}
\newcommand{\cw}{c_w}
\newcommand{\ii}{\mathrm{i}}
\newcommand{\vA}{\mathbf{A}}
\newcommand{\vB}{\mathbf{B}}
\newcommand{\vC}{\mathbf{C}}
\newcommand{\vD}{\mathbf{D}}
\newcommand{\vV}{\mathbf{V}}
\newcommand{\vT}{\mathbf{T}}
\newcommand{\vW}{\mathbf{W}}
\newcommand{\vw}{\mathbf{w}}
\newcommand{\cD}{\mathcal{D}}
\newcommand{\cS}{\mathcal{S}}
\newcommand{\cP}{\mathcal{P}}
\newcommand{\pd}{\partial}
\newcommand{\mfrac}[2]{#1 / #2}
\newcommand{\tr}[1]{\mathop{\rm tr}\left\{#1\right\}}
\newcommand{\LL}{\mathcal{L}}
\newcommand{\MSbar}{\mbox{$\overline{\rm MS}$}}
\newcommand{\pp}{{\prime 2}}
\newcommand{\z}{\phantom{0}}

\newcommand{\polRL}{e^-_R e^+_L}
\newcommand{\polLR}{e^-_L e^+_R}
\newcommand{\ttbar}{ t \bar t}
\newcommand{\bbbar}{ b \bar b}
\newcommand{\Gammat}{\Gamma_t}
\newcommand{\rmsn}{\mathrm{rms}_{90}}
\newcommand{\Pt}{P_T}
\newcommand{\eplus}{e^+}
\newcommand{\eminus}{e^-}
\newcommand{\epem}{\eplus\eminus}
\def\invfb{ \mbox{fb}^{-1} } 
\newcommand{\alr}{A_{LR}}
\newcommand{\afb}{A_{FB}}
\newcommand{\afbt}{A^t_{FB}}
\newcommand{\afbl}{A^{\ell}_{FB}}
\newcommand{\thel}{\theta_{hel}}
\newcommand{\cthel}{\mathrm{cos} \theta_{hel}}
\newcommand{\qq}{q\bar{q}}
\newcommand{\Zzero}{Z^0}
\newcommand{\tpq}{t}
\newcommand{\Wboson}{W}
\newcommand{\bottom}{b}
\newcommand{\quark}{q}
\newcommand{\qll}{Q_{LL}}
\newcommand{\qlr}{Q_{LR}}
\newcommand{\qrl}{Q_{RL}}
\newcommand{\qrr}{Q_{RR}}
\newcommand{\leftp}{\left(}
\newcommand{\rightp}{\right)}
\newcommand{\lhel}{\lambda_t}
\newcommand{\fonevI}{{\cal F}^{I}_{1V}}
\newcommand{\ftwovI}{{\cal F}^{I}_{2V}}
\newcommand{\foneaI}{{\cal F}^{I'}_{1A}}
\newcommand{\ftwoaI}{{\cal F}^{I}_{2A}}
\newcommand{\pem} { {\cal P} }
\newcommand{\pep} {{\cal P'} }
%
%\newcommand{\f1vr}{\cal{F}^{R}_{1V}}
%\newcommand{\f2vr}{\cal{F}^{R}_{2V}}
%\newcommand{\f1ar}{\cal{F}^{R}_{1A}}
%\newcommand{\f2ar}{\cal{F}^{R}_{2A}}
%\rhead{Version 1.3}

%\rhead{
%Top@ILC\\
%White paper for Snowmass CSS 2013}
%\renewcommand{\headrulewidth}{0in}

%\newcommand{\chi2}{\chi^2}


%%%%%%%%%%%%%%%%%%%%%%%%%%%%%%%%%%%%%%%%%%%%%%%%%%%%%%%%%%%%%%%%%%%%%%%%

%%  bibliographic entries 

%    Please use the properly formed citations supplied by the
%    INSPIRES   LaTeX(US) link.  Please use the INSPIRE tags
%    for reference.  A typical bibliograph entry would then be:

%  \bibitem{Tsukamoto:1993gt}
%      T.~Tsukamoto, K.~Fujii, H.~Murayama, M.~Yamaguchi, Y.~Okada,
%      %``Precision study of supersymmetry at future linear e+ e- colliders,''
%       Phys.\ Rev.\  {\bf D51}, 3153-3171 (1995).

%   You would refer to this paper as:    \cite{Tsukamoto:1993gt}
%     duplicate references between chapters will be reconciled by 
%        the  editor in the final version

%%%%%%%%%%%%%%%%%%%%%%%%%%%%%%%%%%%%%%%%%%%%%%%%%%%%%%%%%%%%%%%%%%%%%%%%%
%%%%%%%%%%%%%%%%%%%%%%%%%%%%%%%%%%%%%%%%%%%%%%%%%%%%%%%%%%%%%%%%%%%%%%%%%%%%
%  put any macros needed for this section here

\input mymacros.tex

%  or list them here:

\def\IFIC{\footnote{IFIC, c/ Catedr\`atico Jos\'e Beltr\`an, 2  46980 Paterna, SPAIN} }
\def\KEK{\footnote{KEK, 1-1 Oho, Tsukuba-shi, Ibaraki-ken 305-0801, JAPAN}}
\def\Carleton{\footnote{Ottawa-Carleton Inst. for Physics, 
   Carleton University, Ottawa, Ontario K1S 5B6, CANADA}}
\def\DESY{\footnote{DESY, Notkestrasse 85, D-22607 Hamburg,  GERMANY}}
\def\SLAC{\footnote{SLAC, Stanford 
University, 2575 Sand Hill Rd., Menlo Park, CA 94025, USA}}
\def\UTA{\footnote{Physics Department, University of Texas,
Arlington, TX 76019,  USA}}
\def\Tsinghua{\footnote{Department of Engineering Physics, Tsinghua
    University, Beijing,  CHINA}}
\def\Toyama{\footnote{Department of Physics, University of Toyama, 
        3190 Gofuku, Toyama 930-8555, JAPAN}}
\def\Cornell{\footnote{Department of Physics, Cornell University, Ithaca NY 
                 14853,  USA}}
\def\Vienna{\footnote{Inst. f\"ur Theor. Physik, Universit\"at Wien, 
      Boltzmanngasse 5, A-1090 Vienna, AUSTRIA}}
\def\Oxford{\footnote{Department of Physics, University of Oxford,
         Keble Road, Oxford OX1 3RH, UK}}
\def\Oklahoma{\footnote{University of Oklahoma, Norman, OK 73019, USA}} 
\def\Orsay{\footnote{LAL, Universit\'e Paris Sud, 
           F-91898 Orsay CEDEX, FRANCE}}
\def\ParisD{\footnote{Astroparticle and Cosmology Laboratory,
Universit\'e Paris-Diderot, Paris 7/CNRS, FRANCE}}
\def\Irvine{\footnote{Department of Physics, University of California,
    Irvine, CA  USA}}
\def\CERN{\footnote{TH Division, Case C01600, CERN, CH-1211 Geneva 23,
    SWITZERLAND}}
\def\okla{\footnote{University of Oklahoma, Norman, OK 73019, USA}}
\def\uminn{\footnote{University of Minnesota, Minneapolis, MN 55455, USA}}

%%%%%%%%%%%%%%%%%%%%%%%%%%%%%%%%%%%%%%%%%%%%%%%%%%%%%%%%%%%%%%%%%%%%%%%%%%%

\title{ \LARGE\bf Silicon Detector (SiD) Tracking Geometry Optimization Studies for the ILC}
%\begin{footnotesize}
%***********************************************************************
% AUTHORS INFORMATION AREA
%***********************************************************************
%\author[1]{Author A\thanks{A.A@university.edu}}
%\author[1]{Author B\thanks{B.B@university.edu}}
%\author[1]{Author C\thanks{C.C@university.edu}}
%\author[2]{Author D\thanks{D.D@university.edu}}
%\author[2]{Author E\thanks{E.E@university.edu}}
\author[1]{S.~U.~Setru}
\author[1]{M.~Demarteau\thanks{Corresponding author: demarteau@anl.gov}}
\affil[1]{\footnotesize Argonne National Laboratory, 9700 South Cass Avenue, IL, USA}
%\affil[2]{\footnotesize Argonne National Laboratory, 9700 South Cass Avenue, IL, USA}
\renewcommand\Authands{ and }

%\end{footnotesize}

\begin{document}
%\title{ \LARGE\bf Asymmetries in $t\bar{t}$ production at $\roots=500\,\GeV$ at the International Linear Collider} %% 

%\author{\centering I.~Garcia\IFIC, R.~P\"oschl\Orsay, E.~Ros\IFIC, F.~Richard\Orsay\IFIC,J.~Rou\"en\'e\Orsay}
\date{}
%%***********************************************************************
% END OF AUTHORS INFORMATION AREA
%***********************************************************************

\maketitle
\thispagestyle{fancy}


\begin{abstract}
We studied modifications to the tracking system of 
the baseline Silicon Detector (SiD) geometry for the proposed International Linear Collider (ILC)
as presented in the SiD Detailed Baseline Design (DBD) \cite{Behnke:2013lya}.
%We present the tracking performance of the modified Sidloi3 geometry.
The vertex barrel has five roughly evenly spaced tracking layers.
The spatial arrangement of the vertex barrels was modified
to three sets of
 two closely-spaced tracking layers.
The tracking performance was studied using Monte Carlo simulations 
of single muons and two physics processes, 
$\ee \rightarrow \ttbar$ at $ \sqrt{s} = $ 500 GeV and 
$\ee \rightarrow \ttbar \bbbar$ (hadronic decays only) at $ \sqrt{s} = $ 1 TeV.
The modified detector has greater $z$-axis 
impact parameter resolution for high polar angles ($\theta > 60^{\circ}$)
for single muons at 10 and 100 GeV.
The modified detector has lower efficiency for low energy single muons in the forward region.
In the central region, the modified detector has better $z$-axis impact parameter resolution
but worse transverse impact parameter
resolution for high energy muons.
For the two physics processes, the modified detector has
higher efficiency for low energy particles, lower efficiency
for high energy particles, and lower efficiency as a function of the number
hits produced by the reconstructed Monte Carlo charged particles.
The modified detector also has higher track fake rates
for both physics processes for a wide range
of polar angle and transverse momentum.
In all other aspects of detector performance,
both detectors performed comparably.
Neither detector geometry performed unequivocally better than the other
for the parameter space studied.
%Results indicate that the modified detector had an improved 
%Z-axis impact parameter resolution for single muons at high polar angles.
%The modified detector also demonstrated higher track fake rates for both 
%physics processes for a wide range of polar angle and transverse momentum.
\end{abstract}


%\begin{titlepage}

%\vfill
%\begin{center} 
%{\Large   Physics at the International Linear Collider} 

%\vfill

%Physics Chapter of the ILC Detailed Baseline Design Report 

%\vfill

%Preliminary Version:   Draft of \today 

%\medskip

%please address questions or comments to:   poeschl@lal.in2p3.fr

%\vfill

%Editorial Team:
%\vfill
%I.~Garcia\Oklahoma,
%F.~Richard\SLAC,   
%E.~Ros\KEK,
%M.~Vos\Tsinghua,
%J.~Trenado\Vienna,
%T.~Frisson\Toyama,
%R.~P\"oschl\DESY,
%J.~Rou\"en\'e\ Logan\Carleton,
%Andrei Nomerotski\Oxford,
%J\"urgen Reuter\DESY,
%Maxim Perelstein\Cornell,
% Michael E. Peskin\SLAC,
%Roman Poeschl\Orsay,
%Aurore Savoy-Navarro\ParisD,
%Geraldine Servant\CERN,
%Tim M. P. Tait\Irvine,
%Jaehoon Yu\UTA

%\vfill
%\vfill
%%\end{center}

%\end{titlepage}
\def\thefootnote{\fnsymbol{footnote}}
\setcounter{footnote}{0}
%
%\tableofcontents
%\newpage


\input intro.tex
\input methods.tex
\input results.tex
\input discussionconclusion.tex
%\input crosscoup.tex
%\input top-hadron.tex

%%%% THREE ORIGINALS
%\input top-mass.tex
%\input ilc-coupl.tex
%\input top-remarks.tex



%\input selection.tex
%\input crossalr.tex
%\input afb.tex
%\input hel-asym.tex
%\input sys.tex
%\input interpret.tex
%\input sum.tex
%\input annex.tex
\bibliographystyle{utphys_mod}
\begin{footnotesize}
%\bibliography{sm2013-top}
%\bibliography{/Users/sagarsetru/Documents/mybibfile}
\providecommand{\href}[2]{#2}\begingroup\raggedright\begin{thebibliography}{10}

\bibitem{Behnke:2013lya}
T.~Behnke, J.~E. Brau, P.~N. Burrows, J.~Fuster, M.~Peskin, {\em et al.},
  ``{The International Linear Collider Technical Design Report - Volume 4:
  Detectors}''
\href{http://arxiv.org/abs/1306.6329}{{\tt arXiv:1306.6329 [physics.ins-det]}}.
%%CITATION = ARXIV:1306.6329;%%.

\bibitem{Baer:2013cma}
H.~Baer, T.~Barklow, K.~Fujii, Y.~Gao, A.~Hoang, {\em et al.}, ``{The
  International Linear Collider Technical Design Report - Volume 2: Physics}''
\href{http://arxiv.org/abs/1306.6352}{{\tt arXiv:1306.6352 [hep-ph]}}.
%%CITATION = ARXIV:1306.6352;%%.

\bibitem{2011arXiv1104.4547W}
T.~G. {White}, ``{Performance studies of a pixel tracker in the Silicon
  Detector (SiD) concept for a future linear collider}''{\em ArXiv e-prints}
  (Apr., 2011)  , \href{http://arxiv.org/abs/1104.4547}{{\tt arXiv:1104.4547
  [physics.ins-det]}}.

\bibitem{graf2007simulator}
N.~Graf and J.~McCormick, ``Simulator for the linear collider (slic): a tool
  for ilc detector simulations'' in {\em AIP Conf. Proc. 867: 503-512, 2006},
  no.~SLAC-PUB-12350, Stanford Linear Accelerator Center (SLAC).
\newblock 2007.

\bibitem{agostinelli2003geant4}
S.~Agostinelli, J.~Allison, K.~E. Amako, J.~Apostolakis, H.~Araujo, P.~Arce,
  M.~Asai, D.~Axen, S.~Banerjee, G.~Barrand, {\em et al.}, ``Geant4—a
  simulation toolkit'' {\em Nuclear instruments and methods in physics research
  section A: Accelerators, Spectrometers, Detectors and Associated Equipment}
  {\bf 506} (2003) no.~3, 250--303.

\bibitem{allison2006geant4}
J.~Allison, K.~Amako, J.~Apostolakis, H.~Araujo, P.~A. Dubois, M.~Asai,
  G.~Barrand, R.~Capra, S.~Chauvie, R.~Chytracek, {\em et al.}, ``Geant4
  developments and applications'' {\em Nuclear Science, IEEE Transactions on}
  {\bf 53} (2006) no.~1, 270--278.

\bibitem{lcsimurl}
``Linear collider simulation'' 2014.
\newblock \url{http://www.lcsim.org/sites/lcsim/index.html}.

\bibitem{graf2011org}
N.~A. Graf, ``org.lcsim: Event reconstruction in java'' in {\em Journal of
  Physics: Conference Series}, vol.~331, p.~032012, IOP Publishing.
\newblock 2011.

\bibitem{1742-6596-513-3-032077}
C.~Grefe, S.~Poss, A.~Sailer, A.~Tsaregorodtsev, the Clic~detector, and physics
  study, ``Ilcdirac, a dirac extension for the linear collider community'' {\em
  Journal of Physics: Conference Series} {\bf 513} (2014) no.~3, 032077.
  \url{http://stacks.iop.org/1742-6596/513/i=3/a=032077}.

\bibitem{tsaregorodtsev2008dirac}
A.~Tsaregorodtsev, M.~Bargiotti, N.~Brook, A.~C. Ramo, G.~Castellani,
  P.~Charpentier, C.~Cioffi, J.~Closier, R.~G. Diaz, G.~Kuznetsov, {\em et
  al.}, ``Dirac: a community grid solution'' in {\em Journal of Physics:
  Conference Series}, vol.~119, p.~062048, IOP Publishing.
\newblock 2008.

\bibitem{2003physics...6114G}
F.~{Gaede}, T.~{Behnke}, N.~{Graf}, and T.~{Johnson}, ``{LCIO: A persistency
  framework for linear collider simulation studies}''{\em ArXiv Physics
  e-prints} (June, 2003)  , \href{http://arxiv.org/abs/physics/0306114}{{\tt
  physics/0306114}}.

\bibitem{stdhep:url}
``Stdhep'' 2013.
\newblock \url{http://cepa.fnal.gov/psm/stdhep/}.

\bibitem{kilian2011whizard}
W.~Kilian, T.~Ohl, and J.~Reuter, ``Whizard: simulating multi-particle
  processes at lhc and ilc'' {\em The European Physical Journal C} {\bf 71}
  (2011) no.~9, 1--29.

\bibitem{sjostrand2006pythia}
T.~Sj{\"o}strand, S.~Mrenna, and P.~Skands, ``Pythia 6.4 physics and manual''
  {\em Journal of High Energy Physics} {\bf 2006} (2006) no.~05, 026.

\bibitem{Grefe:2014pba}
C.~Grefe, ``{Detector Optimization Studies and Light Higgs Decay into Muons at
  CLIC}''
\href{http://arxiv.org/abs/1402.2780}{{\tt arXiv:1402.2780 [physics.ins-det]}}.
%%CITATION = ARXIV:1402.2780;%%.

\bibitem{geant4physlist:url}
``Geant4 physics reference manual'' 2011.
\newblock \url{http://cepa.fnal.gov/psm/stdhep/}.

\bibitem{2008arXiv0808.0130P}
S.~{Piperov}, ``{Geant4 validation with CMS calorimeters test-beam data}''{\em
  ArXiv e-prints} (Aug., 2008)  , \href{http://arxiv.org/abs/0808.0130}{{\tt
  arXiv:0808.0130 [physics.ins-det]}}.

\bibitem{krammer1997signal}
M.~Krammer and H.~Pernegger, ``Signal collection and position reconstruction of
  silicon strip detectors with 200 $\mu$m readout pitch'' {\em Nuclear
  Instruments and Methods in Physics Research Section A: Accelerators,
  Spectrometers, Detectors and Associated Equipment} {\bf 397} (1997) no.~2,
  232--242.

\bibitem{kramer2006track}
T.~Kr{\"a}mer, ``Track parameters in lcio'' tech. rep., LC-DET-2006-004, 2006.

\bibitem{alcaraz1995helicoidal}
J.~Alcaraz, ``Helicoidal tracks'' tech. rep., L3 Note 1666, February 1995,
  1995.

\end{thebibliography}\endgroup
\end{footnotesize}

\end{document}


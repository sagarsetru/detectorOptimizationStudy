\section{Concluding remarks}
The $t$~quark could be a window to new physics associated with light 
composite Higgs bosons and strong coupling in the Higgs sector.
The key parameters here are the electro-weak couplings of the $t$~quark.  
We have demonstrated that the ILC offers unique capabilities
to access these couplings and measure them to the required high level of
 precision. The mass of the $t$~quark, which is a most important quantitiy
in many theories can be measured model independent to a precision of 
 better than 100~MeV. It has however to be pointed out that 
all of these precision measurements require a superb detector performance and 
event reconstruction. The key requirements are the tagging of final state
$b$ quarks with and efficiency and purity of better than 90\% and jet
 energy reconstruction using particle flow of about 4\% in the entire
 accessible energy range. These requirements are met for the ILC detectors
described in the detector volume of~\cite{bib:ilc-tdr-dbd}. 

On the other hand the physics program require state-of-the art theoretical calculations for the observables. While QCD corrections seem to be largely under control, future work should, at least w.r.t the form factors, address the uncertainties on the NLO electroweak corrections. For meaningful experimental studies the existing event generators will have to actively supported.  

%Top quark physics at the ILC has Noble Price potential!
{\em The full exploitation of the potential of $t$~quark physics at the ILC requires a very close collaboration between  theoretical and experimental 
groups over the coming years.} The points outlined in this contribution may serve as a basis for the establishment of such a collaboration.
